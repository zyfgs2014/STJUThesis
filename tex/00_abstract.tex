%# -*- coding: utf-8-unix -*-
%%==================================================
%% abstract.tex for SJTU Master Thesis
%%==================================================

\begin{abstract}

随着在线旅游业的发展,在线机票订购业务为用户提供在线选购机票的服务支持。虽然其丰富了用户的选择,但对于一些热门航线,搜索结果中的候选机票的数量过多。增加了用户的选择成本,降低了交互效率。如果能够根据每位用户的偏好,为用户从候选机票中推荐满足用户偏好的机票,可以更有效地克服机票购买中信息过载问题。

论文主要研究了用户偏好建模及机票推荐。结合机票价格动态变化的特征并根据机票选购的实际业务场景,提出了基于内容的方法为用户提供个性化机票推荐。其后,论文分析了机票推荐中仍存在的一些问题,并细化了推荐场景,进一步提升机票推荐的准确率。文章的主要工作体现在以下几个方面:

1.提出基于用户特征分布的机票推荐算法。对数据集进行研究与分析,提取影响用户购买决策的机票特征。根据用户的历史订单构建用户特征分布模型;提出了模拟机票实时推荐场景。本条工作作为深入研究的基础,贯穿所有后续章节。

2.提出解决机票推荐冷启动问题的方法。分析航线特征分布差异对机票推荐准确率的影响,提出机票推荐冷启动问题。总体分为航线冷启动和用户冷启动两种类型。对于前者,根据航线特征分布和用户行为衡量航线间相似度,为用户构建修正混合模型;对于后者,结合共同乘客和同乘关系组成用户间的社会关系,提出增强用户模型的算法。

3.提出结合共享账户乘客预测的机票推荐算法。根据同一账户下乘客的偏好差异,提出了机票推荐中共享账户问题。使用作者-主题模型对乘客的概率分布进行预测。并结合当前会话的上下文信息计算乘客概率分布,将乘客预测结果结合到机票推荐过程中,提供更具针对性的推荐。

4.提出结合隐性特征的机票推荐算法。通过分析不同起飞日期的机票推荐准确率,验证了用户行为变化的事实。建立了转换矩阵将用户模型和机票内容映射到隐性空间。通过参数估计得到用户在隐性空间的偏好模型。综合了用户显性偏好模型和隐性偏好模型,得到最终机票推荐结果。

\textbf{关键词:} 机票推荐,冷启动,主题模型,共享账户,隐性特征
\end{abstract}






\begin{englishabstract}


\englishkeywords{\large SJTU, master thesis, XeTeX/LaTeX template}
\end{englishabstract}

